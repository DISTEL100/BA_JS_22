\newglossaryentry{g:Menge}{
	name=Menge,
	description={
		Eine Zusammenfassung verschiedener Konstanten.
	}
}

\newcommand{\jjset}[1]{\glslink{g:Menge}{\ensuremath{\mathbb{#1}}}}

\newglossaryentry{s:Menge}{
	name=\jjset{A},
	description={Menge},
	sort=menge,
	type=symbolslist
}

\newcommand{\jjpset}[1]{\glslink{g:Menge}{\ensuremath{\mathbb{\widehat{#1}}}}}

\newglossaryentry{s:Potenzmenge}{
	name=\jjpset{A},
	description={Potenzmenge},
	sort=menge_potenz,
	type=symbolslist
}

\begin{defn}[\gls{g:Menge}]
	\glsdesc{g:Menge}
	Eine Konstante, die in eine Menge gefasst wurde heißt \textit{Element der Menge}.
	$0$ Konstanten zu einer Menge zusammengefasst ergibt die \textit{leere Menge}.
\end{defn}

\paragraph{Schreibweise}
Wir schreiben Mengen als einen großen Buchstaben, z.~B. \jjset{A} oder \jjset{B}.
Wollen wir eine Menge \jjset{X} explizit angeben, listen wir ihre Elemente auf und schreiben $\jjset{X} = \{ \jjc{a},\jjc{b},\jjc{c},\jjc{d}\}$.
Genauso gut sind aber alle Beschreibungen, die keine Zweifel zulassen, was zur Menge gehört und was nicht.
Eine \textit{Potenzmenge} \jjpset{A}, die Menge aller Teilmengen von \jjset{A}, schreiben wir mit einem Dach.




