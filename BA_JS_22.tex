% !TeX program = xelatex
\documentclass[leqno, abstracton]{scrreprt}
\usepackage[ngerman]{babel}
\usepackage{import}
\usepackage{color}
\usepackage{hyperref}
\hypersetup{ colorlinks=true, pdfhighlight=/P }
\usepackage[a4paper]{geometry}
\usepackage{fontspec}
\setmainfont[Ligatures=TeX]{IBM Plex Sans Condensed}
\usepackage{amsmath,amsthm}
\usepackage[mathrm=sym]{unicode-math}
\setmathfont{Fira Math}
\usepackage[ngerman]{translator}
\usepackage[ acronym ]{glossaries}
\theoremstyle{plain}% default
\newtheorem{thm}{Theorem}[section]
\newtheorem{lem}[thm]{Lemma}
\newtheorem{prop}[thm]{Proposition}
\newtheorem*{cor}{Corollary}
\newtheorem*{KL}{Klein’s Lemma}

\theoremstyle{definition}
\newtheorem{defn}{Definition}[section]
\newtheorem{conj}{Conjecture}[section]
\newtheorem{exmp}{Example}[section]

\theoremstyle{remark}
\newtheorem*{rem}{Remark}
\newtheorem*{note}{Note}
\newtheorem{case}{Case}

%Neues Glossary für Symbole
\newglossary[slg]{symbolslist}{syi}{syg}{Symbolverzeichnis}

% Glossary entry with index
\glsnoexpandfields

\newcommand*{\glsarg}{}

\defglsentryfmt{%
  \let\orgglsarg\glsarg
  \ifdefempty\glsinsert
  {}%
  {%
    \let\glsarg\glsinsert
    \let\glsinsert\relax
  }%
  \glsgenentryfmt
  \let\glsarg\orgglsarg
}


\makeglossaries
\usepackage{csquotes}
\usepackage{biblatex}
\addbibresource{./refs.bib}
\usepackage{blindtext}
\newfontfamily\firalight{Fira Sans UltraLight}
\newcommand{\ts}[1]{\textsubscript{#1}}



%------------------------------------------------------------------------------
% Begin
%------------------------------------------------------------------------------
\begin{document}

%-Titlepage--------------------------------------------------------------------
\title{Ein Vollständigkeistbeweis für das implikative Fragment der Aussagenlogik im Stil von Henkin}
\author{Jonathan Schneider}
\maketitle


%-Abstract---------------------------------------------------------------------
\renewcommand{\abstractname}{Abstract}
\input{./frontmatter/abstract}

%-Table-of-contents------------------------------------------------------------
\tableofcontents

%-Inhalt-----------------------------------------------------------------------
\chapter{Prelimaries}
\import{chapters}{prelimaries}

\chapter{Sprachen}
\import{chapters}{sprachen}

%-Glossar----------------------------------------------------------------------
\glsaddall
\printglossary[style=altlist,title=Glossar]

%-Abkürzungen------------------------------------------------------------------
\printglossary[type=\acronymtype,style=long]

%-Symbole----------------------------------------------------------------------
\printglossary[type=symbolslist,style=long]

%-Bibliography-----------------------------------------------------------------
\printbibliography

\end{document}
