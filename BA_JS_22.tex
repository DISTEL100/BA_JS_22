% !TeX program = xelatex
\documentclass{scrreprt}

\usepackage[ngerman]{babel}
\usepackage{color}
\usepackage[a4paper]{geometry}
	

\usepackage{fontspec}
\setmainfont[Ligatures=TeX]{IBM Plex Sans Condensed}

\usepackage{amsmath}
\usepackage[mathrm=sym]{unicode-math}
\setmathfont{Fira Math}
\usepackage[ngerman]{translator}

\usepackage{hyperref}
\hypersetup{colorlinks=true,
pdfhighlight=/P
}%

\usepackage[ acronym ]{glossaries}
\newglossaryentry{}{
	name=Platzhalter,
	description={
	}
}

\newglossaryentry{g:Name}{
	name=Name,
	description={
		Ein Signifikant oder eine Struktur aus Signifikanten, die mit \textbf{einem} Signifikat ausgestattet sind. Das Paar aus Name und Bedeutung heißt \gls{g:NameRel}. Ein Name \textbf{bezeichnet} etwas. Ein Name drückt einen Sinn aus.
		}
}

\newglossaryentry{g:NameRel}{
	name=Namenrelation,
	description={
		Die Beziehung eines Namens, zu seiner Bedeutung. 
	}
}

\newacronym{a:SymProp}{SP}{Propositionssymbol}
\newglossaryentry{g:SymProp}{
	name=Propositionssymbol,
	description={
		Bezeichnet Propsitionen
	},
	text=\gls{a:SymProp}
}

\newacronym{MS}{MS}{Microsoft}
\newacronym{CD}{CD}{Compact Disc}
%Eine Abkürzung mit Glossareintrag
\newacronym{AD}{AD}{Active Directory\protect\glsadd{glos:AD}}
 

\newglossary[slg]{symbolslist}{syi}{syg}{Symbolverzeichnis}

\newglossaryentry{s:(}{
name=\ensuremath{ \bigr( \glsarg },
description={Die Kreiszahl.},
sort=symbolpi,
type=symbolslist
}


\makeglossaries

\usepackage{csquotes}
\usepackage{biblatex}
\addbibresource{bib/refs.bib}

\usepackage{blindtext}




\begin{document}
\title{Ein Vollständigkeistbeweis für das implikative Fragment der Aussagenlogik im Stil von Henkin}
\author{Jonathan Schneider}
\maketitle


\renewcommand{\abstractname}{Abstract}
\begin{abstract}
\input{./frontmatter/abstract}
\end{abstract}

\tableofcontents

\chapter{Allgemeines}
\LaTeX{} \cite{latex2e} is a set of macros built atop \TeX{} \cite{texbook}.
In unserem Netzwerk setzen wir auf \gls{AD}. Durch den Einsatz
eines \gls{AD} erreichen wir bei \gls{MS}-Systemen, die mit einer
\gls{glos:AntwD} von \gls{CD} installiert wurden, die beste Standardisierung.
\section{Griechische Symbole}
Berechnungen mit \gls{symb:Pi} ergeben stets ein ungenaues Ergebnis,
denn \gls{symb:Pi} ist eine irrationale Zahl. Weiterhin gibt es noch
\gls{symb:Phi} und \gls{symb:Lambda}.
\subsection{subsection}
\gls{glos:Test} ist ein kleiner Test
\blindtext
\subsubsection{subsubsection}
\blindtext
\paragraph{paragraph}
\Blindtext
\subparagraph{supparagraph}
\blindtext

%Glossar ausgeben
\printglossary[style=altlist,title=Glossar]
%Abkürzungen ausgeben
\printglossary[type=\acronymtype,style=long]
%Symbole ausgeben

\printglossary[type=symbolslist,style=long]

%Bibliographie ausgeben
\printbibliography

\end{document}
