\newglossaryentry{g:Konstante}{
	name=Konstante,
	description={
		Ein \textbf{eindeutiges} Zeichen.	
	}
}

\newcommand{\jjc}[1]{\glslink{g:Konstante}{\ensuremath{\mathbf{#1}}}}.

\newglossaryentry{s:Konstante}{
	name=\jjc{a},
	description={Konstanten},
	sort=konstanten,
	type=symbolslist
}

\begin{defn}[\gls{g:Konstante}]
	\glsdesc{g:Konstante}	
	Es gibt keine Zweifel, für welchen Begriff es steht und welches Ding gemeint ist. 
	\cite[9]{ACIntro}
\end{defn}

	\paragraph{Schreibweise} 
	Für eine Konstante benutzen wir einen kleinen, fett gedruckten Buchstaben z.B. \jjc{a} oder \jjc{g}.
