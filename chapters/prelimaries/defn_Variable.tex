\newglossaryentry{g:Variable}{
	name=Variable,
	description={
		Ein Zeichen, das eine festgelegte Menge an Interpretationen erlaubt. Für Bedeutung und Sinn gibt es also mehrere Möglichkeiten.
	}
}

\newcommand{\jjv}[1]{\glslink{g:Variable}{\ensuremath{#1}}}

\newglossaryentry{s:Variable}{
	name=\jjv{a},
	description={Variable},
	sort=1variable,
	type=symbolslist
}

\begin{defn}[\gls{g:Variable}]
	\glsdesc{g:Variable}
	Eine Variable kann also den Wert einer \textit{beliebigen} Konstanten aus einer Ausgangsmenge annehmen.
	Wollen wir eine zutreffende Aussage über eine Variable machen, so müssten wir sicherstellen, dass die Aussage auf alle Konstanten der Ausgangsmenge zutrifft.
	\cite[9]{ACIntro}
\end{defn}

\paragraph{Schreibweise} Variablen schreiben wir als einfache kleine Buchstaben, z.B. \jjv{v} oder \jjv{x}.
Eine Variable \jjv{n}, die als Ausgangsmenge die Menge \jjset{N} hat, schreiben wir  $\jjv{n} \in \jjset{N}$.
