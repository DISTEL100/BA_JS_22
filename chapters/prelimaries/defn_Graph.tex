\newglossaryentry{g:Graph}{
	name=Graph,
	description={
		Ist die Mengentheoretische Beschreibung einer Funktion.
	}
}

\newcommand{\jjg}[1]{\glslink{g:Graph}{\ensuremath{\mathit{\##1}}}}.

\newglossaryentry{s:Graph}{
	name=\jjg{a},
	description={Graph},
	sort=funktion_graph,
	type=symbolslist
}

\begin{defn}[\gls{g:Graph}]
	\glsdesc{g:Graph}
	Der Graph der Funktion $\jjf{a}: \jjset{B} \mapsto \jjset{C}$ ist die Menge
	$\jjset{A} \subseteq \jjset{B} \times \jjset{C}$,
	die alle Paare $ ( \jjc{b}, \jjc{c} ) $
	enthält, mit $\jjf{a}\jjc{b} = \jjc{c}$
\end{defn}

\paragraph{Schreibweise} Den Graphen der Funktion \jjf{a} schreiben wir \jjg{a}
