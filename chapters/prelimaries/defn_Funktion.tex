\newglossaryentry{g:Funktion}{
	name=Funktion,
	description={
		Ist eine Operation, die Konstanten einer Ausgangsmenge in Konstanten einer Zielmenge überführt.
	}
}

\newcommand{\jjf}[1]{\glslink{g:Funktion}{\ensuremath{\text{{\firalight\textit{#1}}}}}}

\newglossaryentry{s:Funktion}{
	name=\jjf{a},
	description={Funktion},
	sort=funktion,
	type=symbolslist
}

\begin{defn}[\gls{g:Funktion}]
	\glsdesc{g:Funktion}
	Einer Konstante der Ausgangsmenge wird dabei \textit{höchstens} ein Element Zielmenge zugeordnet.
\end{defn}

\paragraph{Schreibweise} Eine Funktion von \jjset{A} nach \jjset{B} schreiben wir mit kleinem, kursiven Buchstaben: $\jjf{a}: \jjset{A} \mapsto \jjset{B}$.
Das Element der Zielmenge, dem \jjf{a} das Element $\jjc{a} \in \jjset{A}$ zuordnet wird \jjc{a}\jjf{a} geschrieben.


