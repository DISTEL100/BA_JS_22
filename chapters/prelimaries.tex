\section{Zeichen}
Semiotisches Dreieck \cite{UESemiotik}\\
Eindeutigkeit \cite{ACIntro} \\ 
arithmetisierung \cite{GFSinn} \\
\newglossaryentry{g:Zeichen}{
	name=Zeichen,
	description={
		Zeichen, Wörter, Gesten, \ldots die einen Begriff erwecken und ein Ding meinen.
		In der Semiotik unterteilt in Ikon, Index und Symbol. \cite{UESemiotik} 
	 }
}

\begin{defn}[\gls{g:Zeichen}]
	Ein Phänomen, das für einen Begriff steht und ein Ding meint.
\end{defn}

\paragraph{}
\newglossaryentry{g:Begriff}{
	name=Begriff,
	description={
		Die Vorstellung eines Dings. Das was im Kopf stattfindet. Das Denken arbeitet mit Begriffen. Bei Frege ist das der Sinn, die Art des Gegebenseins des Dings \cite{GFSinn}.
	}
}

\begin{defn}[\gls{g:Begriff} ]
	Die Vorstellung eines Dings
\end{defn}

\paragraph{}
\begin{defn}[\gls{g:Ding}]
	Ein Ding ist die echte Sache in der Welt.
\end{defn}

\newglossaryentry{g:Ding}{
	name=Ding,
	description={
		Die reale Sache in der Welt. Bei Frege: die Bedeutung von etwas. Der Sinn deutet in die Richtung der Bedeutung \cite{GFSinn} 
	}
}


\section{Konstanten und Variablen}
\newglossaryentry{g:Konstante}{
	name=Konstante,
	description={
		Ein \textbf{eindeutiges} Zeichen.	
		Es gibt keine Zweifel, für welchen Begriff es steht und welches Ding gemeint ist. 
	}
}

\newcommand{\jjc}[1]{\glslink{g:Konstante}{\ensuremath{\mathbf{#1}}}}.

\newglossaryentry{s:Konstante}{
	name=\jjc{a},
	description={Konstanten},
	sort=konstanten,
	type=symbolslist
}

\begin{defn}[\gls{g:Konstante}]
	\glsdesc{g:Konstante}	
	\cite[9]{ACIntro}
\end{defn}

	\paragraph{Schreibweise} 
	Für eine Konstante benutzen wir einen kleinen, fett gedruckten Buchstaben z.B. \jjc{a} oder \jjc{g}.

\paragraph{}
\newglossaryentry{g:Menge}{
	name=Menge,
	description={
		Eine Zusammenfassung \textit{verschiedener} Konstanten.
	}
}

\newcommand{\jjset}[1]{\glslink{g:Menge}{\ensuremath{\mathbb{#1}}}}

\newglossaryentry{s:Menge}{
	name=\jjset{A},
	description={Menge},
	sort=menge,
	type=symbolslist
}

\newcommand{\jjpset}[1]{\glslink{g:Menge}{\ensuremath{\mathbb{\widehat{#1}}}}}

\newglossaryentry{s:Potenzmenge}{
	name=\jjpset{A},
	description={Potenzmenge},
	sort=menge_potenz,
	type=symbolslist
}

\begin{defn}[\gls{g:Menge}]
	\glsdesc{g:Menge}
	Eine Konstante, die in eine Menge gefasst wurde, heißt \textit{Element der Menge}.
	$0$ Konstanten zu einer Menge zusammengefasst ergibt die \textit{leere Menge}.
\end{defn}

\paragraph{Schreibweise}
Wir schreiben Mengen als einen großen Buchstaben, z.~B. \jjset{A} oder \jjset{B}.
Wollen wir eine Menge \jjset{X} explizit angeben, listen wir ihre Elemente auf und schreiben $\jjset{X} = \{ \jjc{a},\jjc{b},\jjc{c},\jjc{d}\}$ oder
genauso gut, geben eine Beschreibung, die keine Zweifel zulässt, ob ein gegebenes Element zur Menge gehört oder nicht.
Eine \textit{Potenzmenge} \jjpset{A}, die Menge aller Teilmengen von \jjset{A}, schreiben wir mit einem Dach.





\paragraph{}
\newglossaryentry{g:Variable}{
	name=Variable,
	description={
		Ein Zeichen, das eine festgelegte Menge an Interpretationen erlaubt. Für Bedeutung und Sinn gibt es also mehrere Möglichkeiten.
	}
}

\begin{defn}[\gls{g:Variable}]
	Ein Zeichen mit einer festgelegten, nichtleeren Menge an möglichen Bedeutungen.	
\end{defn}


\section{Funktionen}
\newglossaryentry{g:Funktion}{
	name=Funktion,
	description={
		Ist eine Operation, die Konstanten einer Ausgangsmenge in Konstanten einer Zielmenge überführt.
	}
}

\newcommand{\jjf}[1]{\glslink{g:Funktion}{\ensuremath{\mathit{#1}}}}.

\newglossaryentry{s:Funktion}{
	name=\jjf{a},
	description={Funktion},
	sort=funktion,
	type=symbolslist
}

\begin{defn}[\gls{g:Funktion}]
	\glsdesc{g:Funktion}
	Einer Konstante der Ausgangsmenge wird dabei \textit{höchstens} ein Element Zielmenge zugeordnet.
\end{defn}

\paragraph{Schreibweise} Eine Funktion von \jjset{A} nach \jjset{B} schreiben wir mit kleinem, kursiven Buchstaben: $\jjf{a}: \jjset{A} \mapsto \jjset{B}$.
Das Element der Zielmenge, dem die Funktion \jjf{a}, die Konstante $\jjc{a} \in \jjset{A}$ zuordnet wird \jjf{a}\jjc{a} geschrieben.


\paragraph{}
Eine spezielle Funktion ist die Relation.
\newglossaryentry{g:Relation}{
	name=Relation,
	description={
		Ist eine Beziehung zwischen Konstanten.
	}
}

\begin{defn}[\gls{g:Relation}]
\glsdesc{g:Relation}
Eine Relation zwischen \jjset{A} und \jjset{B} ist eine Funktion von \jjset{A} nach \jjpset{B}. 
\cite[2]{SEAutom}
\end{defn}

\paragraph{Schreibweise} Eine Relation zwischen \jjset{A} und \jjset{B} schreiben wir wie die Funktionen, mit kleinem, kursiven Buchstaben, aber einem anderen Pfeil: $\jjf{a}: \jjset{A} \twoheadrightarrow \jjset{B}$  

Möchte man eine Beziehung zwischen mehr als zwei Mengen beschreiben,
\paragraph{}
\newglossaryentry{g:Graph}{
	name=Graph,
	description={
		Ist die Mengentheoretische Beschreibung einer Funktion.
	}
}

\newcommand{\jjg}[1]{\glslink{g:Graph}{\ensuremath{\mathit{\##1}}}}.

\newglossaryentry{s:Graph}{
	name=\jjg{a},
	description={Graph},
	sort=funktion_graph,
	type=symbolslist
}

\begin{defn}[\gls{g:Graph}]
	\glsdesc{g:Graph}
	Der Graph der Funktion $\jjf{a}: \jjset{B} \mapsto \jjset{C}$ ist die Menge
	$\jjset{A} \subseteq \jjset{B} \times \jjset{C}$,
	die alle Paare $ ( \jjc{b}, \jjc{c} ) $
	enthält, mit $\jjf{a}\jjc{b} = \jjc{c}$
\end{defn}

\paragraph{Schreibweise} Den Graphen der Funktion \jjf{a} schreiben wir \jjg{a}

